\documentclass[11pt,a4paper]{article}

\usepackage{geometry}
 \geometry{
 a4paper,
 total={150mm,237mm},
 left=30mm,
 top=30mm,
 }

% cf. http://tex.stackexchange.com/questions/50182/subtitle-with-the-maketitle-page
\usepackage{titling}
\newcommand{\subtitle}[1]{%
  \posttitle{%
    \par\end{center}
    \begin{center}\large\textbf{#1}\end{center}
    \vskip0.5em}%
}

\usepackage{color}
\usepackage{graphicx}
\usepackage{subcaption}

\usepackage[utf8]{inputenc}
\usepackage[lf]{venturis} %% lf option gives lining figures as default; 
\usepackage[T1]{fontenc}
\usepackage{beramono}
\usepackage{csquotes}
\usepackage[UKenglish,german]{babel}

\usepackage{fancyvrb}

\widowpenalty10000  % http://tex.stackexchange.com/questions/4152/how-do-i-prevent-widow-orphan-lines
\clubpenalty10000

\title{The SysSon Platform}
\subtitle{Technical Report TR-2016-10-1\\Institute of Electronic Music and Acoustics, Graz\\(Status: in progress)}
\author{Hanns Holger Rutz}
% \date{09-Feb-2016}
\date{October 2016}

% cf. https://tex.stackexchange.com/questions/94126/change-font-to-only-section-and-subsection-of-my-document
%\usepackage{titlesec}
%\titleformat{\chapter}[display]
%  {\fontfamily{pag}\selectfont\huge\bfseries}
%  {\chaptertitlename\ \thechapter}
%  {20pt}
%  {\Huge}
%\titleformat{\section}
%  {\fontfamily{pag}\selectfont\bfseries\Large}
%  {\thesection}
%  {1em}
%  {}
%\titleformat{\subsection}
%  {\fontfamily{pag}\selectfont\bfseries\Large}
%  {\thesection}
%  {1em}
%  {}

\usepackage[backend=biber,authordate]{biblatex-chicago} % citereset=chapter
%\usepackage[backend=biber,natbib,isbn=false,useprefix=true,sorting=ydnt]{biblatex-chicago} % citereset=chapter
\addbibresource{all.bib} % add a bib-reference file
\addbibresource{rutz.bib} % add a bib-reference file

% warning: https://tex.stackexchange.com/questions/313477/
% \usepackage{csquotes}

\usepackage{tabularx}
% cf. https://tex.stackexchange.com/questions/84400/table-layout-with-tabularx-column-widths-502525
\newcolumntype{s}{>{\hsize=1cm}X}

% says you should load after babel and fontspec
\usepackage[shrink=10, babel=true]{microtype}	% http://tex.stackexchange.com/questions/141852/latex-allows-line-break-between-concluding-em-dash-and-comma-before-a-new-sub-cl/141854#141854

% has to come first for full scale TeX voodoo bullcrap
\usepackage{hyperref}
% get rid of the horrible coloured boxes around links
\hypersetup{
    colorlinks,%
    citecolor=black,%
    filecolor=black,%
    linkcolor=black,%
    urlcolor=black
}
% has to come after frickin hyperref
\VerbatimFootnotes

\newcommand{\todo}[1]{\colorbox{yellow}{\textsc{todo}: #1}}

\newcommand{\quot}[1]{\guillemotleft {#1}\guillemotright}

\newcommand{\worktitle}[1]{\textit{#1}}

\newcommand{\workentry}[2]{\vspace{7.5pt}\noindent\textbf{#1} (#2)}
\newcommand{\workentrySel}[2]{\vspace{7.5pt}\noindent\textbf{#1}$*$ (#2)}

\newcommand{\figref}[1]{Fig.~\ref{#1}}

\newcommand{\software}[1]{\textit{#1}}

\newcommand{\sysson}[0]{SysSon}
\newcommand{\syssonVersion}[0]{1.8.0}
\newcommand{\syssonVersionS}[0]{1.8.0-SNAPSHOT}

\newcommand{\artefacts}[0]{\textsc{Artefacts:}}
\newcommand{\assessment}[0]{\textsc{Assessment:}}

\usepackage{listings}

\definecolor{dkgreen}{rgb}{0,0.6,0}
\definecolor{gray}{rgb}{0.5,0.5,0.5}
\definecolor{mauve}{rgb}{0.58,0,0.82}

\lstdefinestyle{plain}{
  frame=tb,
  aboveskip=3mm,
  belowskip=3mm,
  showstringspaces=true,
  columns=flexible,
  basicstyle={\small\ttfamily},
  numbers=none,
  numberstyle=\tiny\color{gray},
  keywordstyle=\color{blue},
  commentstyle=\color{dkgreen},
  stringstyle=\color{mauve},
  frame=none,
  keepspaces=true,
  breaklines=true,
  breakatwhitespace=true,
  tabsize=3,
}

\lstdefinestyle{scala}{
  frame=tb,
  language=scala,
  aboveskip=3mm,
  belowskip=3mm,
  showstringspaces=true,
  columns=flexible,
  basicstyle={\small\ttfamily},
  numbers=none,
  numberstyle=\tiny\color{gray},
  keywordstyle=\color{blue},
  commentstyle=\color{dkgreen},
  stringstyle=\color{mauve},
  frame=none,
  keepspaces=true,
  breaklines=true,
  breakatwhitespace=true,
  tabsize=3,
}

\begin{document}
% \begin{titlepage}
\maketitle
\selectlanguage{UKenglish}
\thispagestyle{empty}
\newpage
\section{First Sonification Scenario}

Following a project meeting on 16-Sep-2016, it was decided that the first scenario or template to work out is to look at QBO (quasi-biennial-oscillation) and ENSI (El Niño), using two sets of measured temperature data.\footnote{%
A different scenario discussed are "stratospheric sudden warmings", which happen around the poles in 30\,km altitude. The sudden jumps in temperature of up to 70 degrees happen within  weeks, and proceed from top to bottom.%
}

We had worked with QBO sonification already in one of the previous workshops. One usually selects a specific range in the altitudes and rather equatorial coordinates (e.g. +/- 10 degrees). Longitudes are often averaged, but we will try to also look at longitudinal movements.

QBO and ENSO interact with each other in that strong amplitudes in ENSO (lower altitudes) are connected with the inverse phenomena in the QBO (higher altitudes). It would thus be interesting to be also able to sonically compare the two.

\subsection{Data Files}

\begin{itemize}
\item {\small \Verb!5x5-climatology_2001-05-01_2016-05-01_RO_OPSv5.6.2_L2b_no_METOP_no_TerraSAR-X.nc!}\\A high-resolution file with 5x5 latitude/longitude grid, 600 altitude levels, and over 180 months (10.1\,GB)
\item {\small \Verb!5x30-climatology_2001-05-01_2016-05-01_RO_OPSv5.6.2_L2b_no_METOP_no_TerraSAR-X.nc!}\\A slightly lower resolution file with 5x50 latitude/longitude grid (1.7\,GB)
\end{itemize}

Initial problems:
%
\begin{itemize}
\item File selection filter did not know about HDF files (previously only NetCDF files were used); \textbf{Result: fixed}
\item Heatmap plots somehow fail to gather the statistics of the data. \textbf{Result: It is just running very slowly (c. 12 minutes).}
\item After stats finally complete, data is useless because we don't have fill value information, and apparently something different from NaN is used. \textbf{Result: Converted file to use NaNs}
\item Date format not recognised. Time unit is "seconds since 1990-01-01T00:00:00", somehow that is not correctly parsed. \textbf{Result:  fixed}
\item The dimensions "Latitude" and "Longitude" are not correctly recognised, because the map-overlay option is not shown. \textbf{Result: fixed}
\end{itemize}
%
We might:
%
\begin{itemize}
\item look into updating the NetCDF library and drop Java 6 support. New artifact is called \Verb!netcdf4! and latest version is 4.6.6. This might improve performance. \textbf{Result: no speed improvement}
\item introduce a simple table view, so one can quickly browse the data numerically.
\item create a preference item to \emph{disable} automatic removal of cache upon application quit.
\item we also need to be able to adjust the cache size, because the 1.7 GB files produces already a 230 KB stats cache, therefore we will easily transcend the default cache size of 1 MB.
\end{itemize}
%
After further inspection, fill values are correctly stored and found as -1e-10. Strangely stats show a max of 2.4e22 and a mean of 5.1e14. For example, for dry temperature those values appear in time index 90 and altitude indices 5 to 8. Here a "spike" of 4.49e23 in altitude index 5 declines towards index 8, and at index 9 the data is in the normal range again. We thus need to first preprocess that file and replace out-of-range values with the defined fill value. \textbf{Result: fixed by converting data.}

\todo{continue here}

\begin{figure}
% \centering
\begin{subfigure}[b]{1.0\textwidth}%
\centering
\includegraphics[scale=0.45,trim=2mm 0 1mm 1mm]{figures/ta_anomalies_mean.jpg}
\caption{Using monthly mean}
\label{fig:anomaly-mean}
\end{subfigure}
\begin{subfigure}[b]{1.0\textwidth}%
\centering
\includegraphics[scale=0.45,trim=2mm 0 1mm -4mm]{figures/ta_anomalies_median.jpg}
\caption{Using monthly median}
\label{fig:anomaly-median}
\end{subfigure}
\caption{Temperature anomalies calculated with different norms.}
\label{fig:anomaly-calc}
\end{figure}

Plots: 105 degrees west, 2.5 degrees south.

First model - time is mapped to time. We use a longitude and latitude index, and a slice of the altitudes. Expected periodicity is 28 to 29 months.

Questions:
%
\begin{itemize}
\item How do we ``connect trajectories'' and avoid steps (in time, because the altitude has a fine resolution)?
\item How do we distinguish too-cold and too-hot? We could in any case project them onto left and right channel, respectively.
\item Altitude resolution is high, we select roughly 200 bands.
\end{itemize}
%

\begin{figure}
\begin{lstlisting}[style=scala]
val vr     = Var("anom")
val dTime  = Dim(vr, "time")
val dAlt   = Dim(vr, "altitude")

val speed  = UserValue("speed", 1).kr
val tp     = dTime.play(speed)
val vp     = vr.play(tp, interp = 2)

val hot    = vp > 1
val amp    = hot * (vp.min(8) / 8)
val alt    = vp.axis(dAlt).values
val altMin = Reduce.min(alt)
val altMax = Reduce.max(alt)

val freq   = alt.linexp(altMin, altMax, 200, 4000)
val sin    = Mix.mono(SinOsc.ar(freq) * amp) / dAlt.size

output := Seq(DC.ar(0), sin)
\end{lstlisting}
\caption{First sketch; filtering only high temperatures,
mapping them each to an oscillator.}
\label{fig:sonif-1-1}
\end{figure}

\begin{figure}
\begin{lstlisting}[style=scala]
val vr    = Var("anom")
val dTime = Dim(vr, "time")
val dAlt  = Dim(vr, "altitude")

val speed = UserValue("speed", 1).kr
val tp    = dTime.play(speed)
val vp    = vr.play(tp, interp = 1)

val max   = ArrayMax.ar(vp)
val freq0 = max.index.linexp(0, dAlt.size - 1, 200, 4000)
val freq  = Ramp.ar(freq0, 1.0/speed)
val amp0  = max.value.clip(0, 8) / 8
val amp   = Ramp.ar(amp0, 1.0/speed)
val sin   = SinOsc.ar(freq) * amp

output := Seq(DC.ar(0), sin)
\end{lstlisting}
\caption{Second sketch; reducing to one oscillator at maximum bin.}
\label{fig:sonif-1-2}
\end{figure}

\begin{figure}
\begin{lstlisting}[style=scala]
...
val max       = ArrayMax.ar(vp)
val maxIdx    = max.index
val numAlt    = dAlt.size
val maskWidth = numAlt / 4
val mask      = vp * ChannelIndices(vp).absdif(maxIdx) > maskWidth
val max2      = ArrayMax.ar(mask)
val maxIdx2   = max2.index
val freq0     = maxIdx.linexp(0, numAlt - 1, 200, 4000)
val freq      = Ramp.ar(freq0, 1.0/speed)
val amp0      = max.value.clip(0, 8) / 8
val amp       = Ramp.ar(amp0, 1.0/speed)
val sin       = SinOsc.ar(freq) * amp

val freq02    = maxIdx2.linexp(0, numAlt - 1, 200, 4000)
val freq2     = Ramp.ar(freq02, 1.0/speed)
val amp02     = max2.value.clip(0, 8) / 8
val amp2      = Ramp.ar(amp02, 1.0/speed)
val sin2      = SinOsc.ar(freq2) * amp2
...
\end{lstlisting}
\caption{Using masking to find multiple ``maxima''}
\label{fig:sonif-multi-max}
\end{figure}

\begin{figure}%
% \thispagestyle{empty}%
\begin{lstlisting}[style=scala]
val numTraj     = 2	// number of trajectories followed
val numVoices   = numTraj * 2
val stateIn     = LocalIn.kr(Seq.fill(numVoices * 2)(0))
var voiceFreq   = Vector.tabulate(numVoices)(i => stateIn \ i): GE
var voiceOnOff  = Vector.tabulate(numVoices)(i => stateIn \ (i + numVoices)): GE
val voiceNos    = 0 until numVoices: GE
Trace(voiceFreq , "vc-freq-in")
Trace(voiceOnOff, "vc-on  -in")
val freqIn      = "freq".kr(Vector.fill(numTraj)(0))
val ampIn       = "amp" .kr(Vector.fill(numTraj)(0))
Trace(freqIn, "in-freq")
Trace(ampIn , "in-amp")
val maxDf       = "max-df".kr(100f)
var activated   = Vector.fill(numVoices)(0: GE): GE

val noFounds = (0 until numTraj).map { tIdx =>
  val fIn       = freqIn \ tIdx
  val aIn       = ampIn  \ tIdx
  val isOn      = aIn > 0
  val freqMatch = (maxDf - (voiceFreq absdif fIn)).max(0)
  val bothOn    = voiceOnOff & isOn
  val bestIn    = 0 +: (freqMatch * (bothOn & !activated))
  val best      = ArrayMax.kr(bestIn)
  val bestIdx   = best.index - 1
  val bestMask  = voiceNos sig_== bestIdx
  activated    |= bestMask
  voiceFreq     = voiceFreq * !bestMask + fIn * bestMask
  Trace(bestIdx, s"f-match $tIdx")
  bestIdx sig_== -1
}

for (tIdx <- 0 until numTraj) {
  val fIn   = freqIn \ tIdx
  val aIn   = ampIn  \ tIdx
  val isOn  = aIn > 0
  val notFound  = noFounds(tIdx)
  val startTraj = notFound & isOn
  val free      = ArrayMax.kr(0 +: (startTraj & !activated))
  val freeIdx   = free.index - 1
  val freeMask  = voiceNos sig_== freeIdx
  activated    |= freeMask
  voiceFreq     = voiceFreq * !freeMask + fIn * freeMask
  Trace(freeIdx, s"f-free  $tIdx")
}

voiceOnOff = activated  // release unused voices
Trace(voiceFreq , "vc-freq-out")
Trace(voiceOnOff, "vc-on  -out")
val stateOut = Flatten(voiceFreq ++ voiceOnOff)
LocalOut.kr(stateOut)
\end{lstlisting}%
\caption{First sketch for voice management}%
\label{fig:source-voice-traces}%
\end{figure}

\begin{figure}
\begin{lstlisting}[style=plain]
---------------------------
control-rate data: 8 frames
---------------------------
f-free  0      : -1.0   -1.0   -1.0   -1.0    0.0   -1.0    0.0   -1.0    
f-free  1      : -1.0   -1.0   -1.0   -1.0   -1.0   -1.0   -1.0   -1.0    
f-match 0      : -1.0   -1.0   -1.0   -1.0   -1.0    0.0   -1.0    0.0    
f-match 1      : -1.0   -1.0   -1.0   -1.0   -1.0   -1.0   -1.0   -1.0    
in-amp      [0]:  0.0    0.0    0.0    0.0    1.0    1.0    1.0    1.0    
in-amp      [1]:  0.0    0.0    0.0    0.0    0.0    0.0    0.0    0.0    
in-freq     [0]:  0.0    0.0  300.0  300.0  300.0  300.0  500.0  500.0    
in-freq     [1]:  0.0    0.0    0.0    0.0    0.0    0.0    0.0    0.0    
vc-freq-in  [0]:  0.0    0.0    0.0    0.0    0.0  300.0  300.0  500.0    
vc-freq-in  [1]:  0.0    0.0    0.0    0.0    0.0    0.0    0.0    0.0    
vc-freq-in  [2]:  0.0    0.0    0.0    0.0    0.0    0.0    0.0    0.0    
vc-freq-in  [3]:  0.0    0.0    0.0    0.0    0.0    0.0    0.0    0.0    
vc-freq-out [0]:  0.0    0.0    0.0    0.0  300.0  300.0  500.0  500.0    
vc-freq-out [1]:  0.0    0.0    0.0    0.0    0.0    0.0    0.0    0.0    
vc-freq-out [2]:  0.0    0.0    0.0    0.0    0.0    0.0    0.0    0.0    
vc-freq-out [3]:  0.0    0.0    0.0    0.0    0.0    0.0    0.0    0.0    
vc-on  -in  [0]:  0.0    0.0    0.0    0.0    0.0    1.0    1.0    1.0    
vc-on  -in  [1]:  0.0    0.0    0.0    0.0    0.0    0.0    0.0    0.0    
vc-on  -in  [2]:  0.0    0.0    0.0    0.0    0.0    0.0    0.0    0.0    
vc-on  -in  [3]:  0.0    0.0    0.0    0.0    0.0    0.0    0.0    0.0    
vc-on  -out [0]:  0.0    0.0    0.0    0.0    1.0    1.0    1.0    1.0    
vc-on  -out [1]:  0.0    0.0    0.0    0.0    0.0    0.0    0.0    0.0    
vc-on  -out [2]:  0.0    0.0    0.0    0.0    0.0    0.0    0.0    0.0    
vc-on  -out [3]:  0.0    0.0    0.0    0.0    0.0    0.0    0.0    0.0    
\end{lstlisting}%
\caption{Trace debug dump for eight control blocks, using the given sequence of values of \emph{in-freq} and \emph{in-amp} for the two trajectories. At the \textbf{fifth} step, the first trajectory's amplitude is set to greater than zero, resulting in an allocation of the first voice. In the \textbf{sixth} step, the frequency matching finds correspondence between first trajectory and first voice, thus preserving the first voice. In the \textbf{seventh}, the first trajectory's frequency jumps from 300\,Hz to 500\,Hz. In the first iteration, no matching trajectory for the first voice is found, making it available again, and thus the \emph{f-free} indicates that it is picked again for the ``new'' trajectory. What we will eventually need is a \emph{release} phase for that voice, before it can be allocated again.}%
\label{fig:trace-dump-voices}%
\end{figure}

%\begin{figure}[h]
%\centering
%\includegraphics[scale=0.5]{figures/ta_anomalies_median.jpg}
%\caption{Temperature anomalies calculated based on monthly median.}
%\label{fig:anomaly-mean}
%\end{figure}

% \printbibliography

\end{document}